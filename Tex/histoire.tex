\section{La carte graphique}
\subsection{Définition}
Une carte graphique, ou carte vidéo, est un périphérique permettant à un ordinateur de communiquer 
avec un écran.\\
\subsection{Composants}
\textbf{GPU} (Graphical Processing Unit) : Processeur graphique, constituant le cœur de la carte graphique, et qui possède des instructions évoluées de traitement d’image. Un GPU est une unité de calcul massivement parallèle.\\\\
\textbf{Mémoire vidéo} (Frame Buffer) : Conserve les images traitées par le GPU avant l’affichage.
\subsection{Fonctionnement en mode texte}
Les premières cartes graphiques datent du début des années 1980, à une époque à laquelle les ordinateurs n'affichaient que du texte à l'écran.\\
Ces cartes ne permettaient d'afficher à l'écran qu'une grille de caractères 
(25 lignes de 80 caractères) prédéfinis qui mesuraient 9x14 pixels chacun. Il était ainsi impossible de modifier directement la valeur d'un pixel.\\
Ce mode de fonctionnement ainsi que la table des caractères utilisables est 
définie par la norme MDA, "Monochrome Display Adapter"\footnote{http://www.seasip.info/VintagePC/mda.html}
du nom de la carte graphique d'IBM qui inaugura cette technologie.\\
Il est à noter que c'est le CPU\footnote{Central Processing Unit : Unité de calcul centrale / Processeur de la machine} qui donnait ses instructions à la carte graphique, celle-ci ne faisant que transmettre les caractères à l'écran.\\
Cette norme est encore utilisée de nos jours, elle permet notamment au BIOS d'afficher des informations au démarrage d'un ordinateur.

\subsection{Fonctionnement en mode graphique}

En 1981 apparait la première carte graphique permettant d'adresser chaque pixel de l'écran indépendamment.
Fabriquée par IBM, cette carte dite CGA, "Color Graphic Adapter" permettait d'utiliser une résolution de 320 par 200 pixels en mode 4 couleurs ou une résolution de 640 par 200 pixels en mode monochrome.\\
Cette carte est une avancée majeure puisqu'elle permet désormais d'afficher n'importe qu'elle forme à l'écran, et plus uniquement des caractères. C'est le début de l'informatique graphique.

\subsection{L'accélération matérielle 2D}

A l'époque, le rôle de la carte graphique se limitait à servir d'intermédiaire entre le CPU et l'écran, c'était le rôle du CPU de définir l'ensemble des pixels à afficher. Par exemple si une application souhaitait tracer une ligne entre deux points A et B, le CPU devait alors calculer la position de chaque pixel composant la ligne avant de demander à la carte graphique d'afficher ceux-ci.\\
Aussi durant les années 1980, avec l'arrivée des interfaces graphiques, les cartes graphiques devinrent de plus en plus performantes dans le but de délester le CPU : elles étaient désormais capables de tracer elles-mêmes des primitives géométriques simples telles que des lignes, des triangles, des rectangles, des cercles, voire de colorier celles-ci d'après les consignes données par le CPU.\\

\subsection{Historique des GPU}
\begin{center}
\begin{tabular}{|c|c|m{0.2\linewidth}|m{0.3\linewidth} |c|}
\hline
Année & Génération & Carte & Application & Bus \\
\hline
1996 & 1 & 3dfx Voodoo & texture mapping et z-buffer & bus PCI\\
\hline
1998 & 2 & GeForce/ Radeon 7500 & Transform\&lighting, multi-texting & bus AGP \\
\cline{1-4}
2001 & 3 & GeForce3/ Radeon 8500 & Programmation sur les sommets (vertex shader)	& \\
\cline{1-4}
2002 & 4 & Radeon 9700/GeForce FX & Programmation sur les pixels (fragment shader)	& \\
\hline
2008 & 5 & GeForce9/ Radeon HD & Compatibilité OpenGL et DirectX,  geometry shader & bus PCIe \\
\hline
\end{tabular}
\end{center}

Les bus PCI(Peripheral Component Interconnect), AGP(Advanced Graphics Port) et  PCIexpress sont des bus local.
Bus local : système de communication entre des cartes d’extension et la carte mère.
Le bus PCIe est une version plus petite et plus performante que le PCI et AGP.
