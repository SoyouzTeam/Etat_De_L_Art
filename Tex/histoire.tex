\section{Historique}

Une carte graphique, ou carte vidéo, est un périphérique permettant à un ordinateur de communiquer 
avec un écran.
Les premières cartes graphiques des années 1980 ne permettaient d'afficher à l'écran qu'une grille de caractères (25 lignes de 80 caractères)
prédéfinis qui mesuraient 9x14 pixels chacun, il était ainsi impossible de modifier directement la valeur d'un pixel.
Ce mode de fonctionnement ainsi que la table des caractères utilisables est définie par la norme "Monochrome Display Adapter"\footnote{http://www.seasip.info/VintagePC/mda.html}
du nom de la première carte graphique d'IBM à utiliser cette éthode.
