% MATLABOUNET
\section{Introduction}
MatLab ou Matrix Laboratory est un logiciel développé par la société The MathWorks, sa fonctionnalité principale est d'effectuer des calculs numériques.\\ 
Il a initialement été conçu pour faciliter le traitement des matrices, cependant il est maintenant utilisé dans tous les domaines des sciences qui nécessite de faire des calculs (par exemple le traitement d'images).\\
Dans sa dernière version MatLab peut utiliser des fonctions d'accélération par le GPU.\\


\section{Les avantages et inconvénients}
\subsection{Avantages}
\begin{itemize}
\item Temps de programmation très rapide pour les calculs et l'affichage
\item Possibilité d'inclure un programme C ou C++
\item C'est un langage interprété, donc il n'y a pas de compilation
\item Une librairie et une aide très bien faite et très riche
\item Code ainsi que fonctions facilement lisibles et compréhensibles
\end{itemize}

\subsection{Inconvénients}
\begin{itemize}
\item Vitesse de calcul beaucoup moins rapide qu'en C ou C++
\item Il est payant
% A vérifier 
\item Application auto-exécutable peu pratique
\end{itemize}

\section{Conclusion}
MatLab est donc généralement utilisé pour faire des expériences en peu de temps. Certaines applications qui nécessiteraient une journée de programmation en C ou C++ peuvent se réaliser en une heure sous MatLab. En revanche, une fois programmée, le temps de calcul sous MatLab peut être cent fois supérieur à celui de C ou C++. De ce fait, MatLab ne s'utilise que très peu pour réaliser un produit finit destiné à un particulier. 

\section{Sources}
http://www.nvidia.fr/object/tesla-matlab-accelerations-fr.html
