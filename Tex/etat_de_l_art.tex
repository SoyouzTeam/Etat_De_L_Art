\documentclass{report}
\usepackage[utf8]{inputenc}
\usepackage[T1]{fontenc}
\usepackage{graphicx}
\usepackage[francais]{babel} 
\usepackage{array} 
\usepackage{hyperref}%pour faire des ancres Table des matières -> partie A | partie B | etc

\title{Etat de l'art}
\author{Projet DUT INFO}
\date{17 Septembre 2013}

\begin{document}
\maketitle

\hypertarget{tableofcontents}{} %affecte des ancres aux differentes partie
\tableofcontents

\part{Introduction}

\chapter{Fonctionnement de l'affichage sur ordinateur}
\section{La carte graphique}
\subsection{Définition}
Une carte graphique, ou carte vidéo, est un périphérique permettant à un ordinateur de communiquer 
avec un écran.\\
\subsection{Fonctionnement en mode texte}
Les premières cartes graphiques datent du début des années 1980, à une époque à laquelle les ordinateurs n'affichaient que du texte à l'écran.\\
Ces cartes ne permettaient d'afficher à l'écran qu'une grille de caractères 
(25 lignes de 80 caractères) prédéfinis qui mesuraient 9x14 pixels chacun. Il était ainsi impossible de modifier directement la valeur d'un pixel.\\
Ce mode de fonctionnement ainsi que la table des caractères utilisables est 
définie par la norme MDA, "Monochrome Display Adapter"\footnote{http://www.seasip.info/VintagePC/mda.html}
du nom de la carte graphique d'IBM qui inaugura cette technologie.\\
Il est à noter que c'est le CPU\footnote{Central Processing Unit : Unité de calcul centrale / Processeur de la machine} qui donnait ses instructions à la carte graphique, celle-ci ne faisant que transmettre les caractères à l'écran.\\
Cette norme est encore utilisée de nos jours, elle permet notamment au BIOS d'afficher des informations au démarrage d'un ordinateur.
\newpage
\subsection{Fonctionnement en mode graphique}

En 1981 apparait la première carte graphique permettant d'adresser chaque pixel de l'écran indépendamment.
Fabriquée par IBM, cette carte dite CGA, "Color Graphic Adapter" permettait d'utiliser une résolution de 320 par 200 pixels en mode 4 couleurs ou une résolution de 640 par 200 pixels en mode monochrome.\\
Cette carte est une avancée majeure puisqu'elle permet désormais d'afficher n'importe qu'elle forme à l'écran, et plus uniquement des caractères. C'est le début de l'informatique graphique.

\begin{center}
\includegraphics[width=10cm,height=8cm]{img/cpuRaster.png}
\includegraphics[width=15cm,height=4cm]{img/cpuRasterExemple.png}
\end{center}
\newpage

\subsection{L'accélération matérielle 2D}

A l'époque, le rôle de la carte graphique se limitait à servir d'intermédiaire entre le CPU et l'écran, c'était le rôle du CPU de définir l'ensemble des pixels à afficher. Par exemple si une application souhaitait tracer une ligne entre deux points A et B, le CPU devait alors calculer la position de chaque pixel composant la ligne avant de demander à la carte graphique d'afficher ceux-ci.\\
Aussi durant les années 1980, avec l'arrivée des interfaces graphiques, les cartes graphiques devinrent de plus en plus performantes dans le but de délester le CPU : elles étaient désormais capables de tracer elles-mêmes des primitives géométriques simples telles que des lignes, des triangles, des rectangles, des cercles, voire de colorier celles-ci d'après les consignes données par le CPU.\\

\begin{center}
\includegraphics[width=9cm,height=8cm]{img/gpuRaster.png}
\includegraphics[width=15cm,height=8cm]{img/gpuRasterExemple.png}
\end{center}
\newpage

\subsection{L'accélération matérielle 3D}
Au début des années 1990 apparaissent les premières applications mettant en œuvre un rendu 3D. C'est notamment le jeu vidéo DOOM datant de décembre 1993 qui est considéré comme étant un pionnier en la matière.\\
L'affichage en 3D nécessite beaucoup de calculs de la part du CPU, celui-ci doit désormais calculer la projection sur l'écran de chaque point de l'espace à rendre.\\
Le constructeur 3DFX inaugure alors la première carte d'accélération 3D : la Voodoo, celle-ci permet de calculer les projections des points à la place du CPU.\\
Au fil des ans, les cartes graphiques deviennent capables d'effectuer la majeure partie des calculs nécessaires à l'affichage,  comme la gestion de l'éclairage, des ombres, l'application des textures, etc. Elles permettent
également d'accélérer le décodage de flux vidéos compressés.\\
On parle désormais de GPU, "Graphics Processing Unit" pour désigner l'unité de calcul d'une carte graphique.\footnote{A la différence du CPU, le GPU dispose d'une architecture massivement parallèle.}

\subsection{Historique des cartes graphiques}
\begin{center}
\begin{tabular}{|c|c|m{0.2\linewidth}|m{0.3\linewidth} |c|}
\hline
Année & Génération & Carte & Application & Bus \\
\hline
1996 & 1 & 3dfx Voodoo & Première carte accélératrice : Texture mapping, Gestion du Z-Buffer & bus PCI\\
\hline
1998 & 2 & GeForce/ Radeon 7500 & Transform\&lighting, multi-texting & bus AGP \\
\cline{1-4}
2001 & 3 & GeForce3/ Radeon 8500 & Programmation sur les sommets (vertex shader)	& \\
\cline{1-4}
2002 & 4 & Radeon 9700/GeForce FX & Programmation sur les pixels (fragment shader)	& \\
\hline
2008 & 5 & GeForce9/ Radeon HD & Compatibilité OpenGL et DirectX,  geometry shader & bus PCIe \\
\hline
\end{tabular}
\end{center}

Les bus PCI,(Peripheral Component Interconnect), AGP(Advanced Graphics Port) et PCIExpress sont des bus local.
Bus local : système de communication entre des cartes d’extension et la carte mère.
Le bus PCIe est une version plus petite et plus performante que le PCI et AGP.

\subsection{Composants d'une carte graphique moderne}
\textbf{GPU} (Graphical Processing Unit) : Processeur graphique, constituant le cœur de la carte graphique, et qui possède des instructions évoluées de traitement d’image. Un GPU est une unité de calcul massivement parallèle.\\\\
\textbf{Mémoire vidéo} (Frame Buffer) : Conserve les images traitées par le GPU avant l’affichage.

\section{Différence entre le CPU et le GPU}
\textbf{\\CPU} (Central Processing Unit):
\begin{itemize}
	\item	Traite l'ensemble des données.
	\item	Données accessibles sans contraintes.
	\item	Format de sortie libre.
	\item	Peu de cœurs pour une exécution  en série
\end{itemize}

\textbf{\\GPU} (Graphics Processing Unit)
\begin{itemize}
	\item	Transforme un flux de données en un autre.
	\item	Format imposé en entrée et en sortie.
	\item	Permet de soulagé le CPU
	\item	Beaucoup de cœurs pour des calculs parallélisés => Accélération d’un code lourds en ressources de calcul.
\end{itemize}

\textbf{\\Les architectures :}
\\
\begin{center}
\includegraphics[width=14cm]{leo/images/GPUCPU.png}
\end{center}

\textbf{\\Conclusion} : Pour que l’utilisation d’un GPU soit utile, il faut qu'il ait suffisamment de coeur (ALU), et que le nombre de calculs soit élevé et parallélisable.


\part{Outils de rendu 3D}

\chapter{OpenGL}

\input{opengl/introduction}%Include section Intro

\section{Fonctionnement statique}
\begin{center}
	 \includegraphics[height=11cm]{img/Fonctionnement}
 \end{center}

La première étape est de définir les primitives des objets à dessiner (\textit{x,y,z} pour chaque Vertex).
OpenGL dessine une image tampon qui sera soit conservée dans la mémoire vidéo de la fenêtre graphique avant d’être affichée, soit une image tampon intermédiaire, on parle alors de double buffering.

La transformation du point de vue sert à la position du plan image. Elle prend en compte la position de la caméra pour se placer correctement autour de l'objet. Ces modifications sont faites à l'aide de matrices. 

Les primitives sont ensuite projetées sur ce plan en fonction des paramètres que l’on a affecté à la projection. Cette projection peut être spécifiée de deux manières. (Voir Projection)

Au final l’image que l’on obtient est redimensionnée en fonction de la taille de la fenêtre graphique . On parle maintenant de pixels et plus de vertex.
\\\\
Source : \cite{OpenGL}
\newpage

\subsection{Primitives}
Tout d'abord, il s'agit de définir chaque objet à modéliser, grâce aux primitives précédemment décrites.

Dans le code, chaque primitive est décrite de la manière suivante:


%//////////////////////////////////
%NE PAS TOUCHER A CE TRUC IMMONDE :)
\begin{tabbing}
XXXX\=XXXX\= \kill

\> \verb|glBegin("type_de_primitive");| \\
\> \> glVertex(\textit{x,y,z});\\
\> \> . \\
\> \> . On définit chaque vertex en fonction du type de primitive\\
\> \> . \\
\> \> glVertex(\textit{x,y,z});\\
\> glEnd();
\end{tabbing}
%NE PAS TOUCHER A CE TRUC IMMONDE :)
%//////////////////////////////////

\subsection{Transformation de point de vue}
La transformation de point de vue consiste en la modification d'une matrice appelée MODELVIEW. Il faut d'abord signaler à OpenGL que l'on veut modifier cette matrice avec la fonction suivante :

\begin{tabbing}
XXXX\= \kill
\> \verb|glMatrixMode( GL_MODELVIEW );|
\end{tabbing}

Ensuite, on charge la matrice identité, ce qui permet de réinitialiser la matrice en cours, puis on définit la caméra.

\begin{tabbing}
XXXX\= \kill
\> \verb|glLoadIdentity( );| \\
\> \verb|gluLookAt(eyeX,eyeY,eyeZ,centerX,centerY,centerZ,upX,upY,upZ);|
\end{tabbing}

La définition de la caméra nécessite 9 paramètres : les 3 premiers pour placer la caméra dans l'espace (\textit{x,y,z}), les 3 suivants pour définir le point regardé par la caméra (\textit{x,y,z}) et les trois derniers pour définir quel est l'axe vertical, en général \textit{y} (On place seulement l'axe voulu à 1 et les autres à 0).\\\\

Il est possible d'effectuer d'autres opérations sur cette matrice, comme une rotation autour d'un des 3 axes ou une translation de vecteurs grâce aux deux fonctions suivantes :

\begin{tabbing}
XXXX\= \kill
\> \verb|glRotatef(angle,x,y,z);| \\\\
Avec l'angle souhaité et \textit{x, y} et \textit{z} à 0 ou 1 suivant autour de quel axe on souhaite\\ effectuer la rotation.\\\\
\> \verb|glTranslatef(vX,vY,vZ);|\\\\
Avec vX, vY et vZ les composantes de la translation. 
\end{tabbing}

\subsection{Projection}

La visualisation d'une scène est réalisée par deux types de transformation :

\begin{itemize}
	\item la première qui se situe dans l'espace 3D et qui permet de positionner le point de vue et si nécessaire les éclairages, pour un rendu plus réaliste (la matrice de modélisation $GL\_MODELVIEW$) que nous avons vu précédemment.

	\item la seconde qui consiste à projeter la scène 3D précédemment construite sur un plan en 2D. Sous OpenGL on retrouve deux formes de projection, la \textbf{projection perspective} et la \textbf{projection orthographique}.
\end{itemize}

La projection appliquée suite aux coordonnées des primitives est définie dans OpenGL grâce à une matrice 4x4. L'usage de ces coordonnées permet de représenter les deux types de projections sous la forme de matrices. \\\\
Source : \cite{PdfOpenGL}

\subsubsection{Perspective}
\begin{center}
	 \includegraphics[height=5cm]{img/Perspective}
 \end{center}
Notre œil perçoit un objet qui est loin plus petit qu’il ne l’est en réalité. Cette méthode de projection est souvent utilisée en animation, simulation et autres applications où il faudrait un degré de réalisme certain, c'est-à-dire, comme peut le percevoir notre œil. Pour la définir il faut connaître son centre de projection (\textit{frustum}), la distance entre ce point et le début de la zone de rendu et entre ce point et la fin de la zone de rendu. Sous OpenGL la fonction est : 
\begin{tabbing}
XXXX\= \kill
\> \verb|glMatrixMode(GL_PROJECTION);|\\
\> \verb|glLoadIdentity();|\\
\> \verb|glFrustum(gauche, droite, bas, haut, proche, éloigné);| \\où les paramètres sont de type double.
\end{tabbing}


\subsubsection{Orthographique}
\begin{center}
	 \includegraphics[height=7cm]{img/Ortho}
 \end{center}
Dans ce cas précis, l’image doit refléter les mesures de l’objet plutôt que son aspect réel. Contrairement à la projection en perspective la taille du \textit{viewing volume} ne change pas, la distance depuis la caméra n’affecte pas la mesure de l’objet. Sous OpenGL la fonction est : 
\begin{tabbing}
XXXX\= \kill
\> \verb|glMatrixMode(GL_PROJECTION);|\\
\> \verb|glLoadIdentity();|\\
\> \verb|glOrtho(gauche, droite, bas, haut, proche, éloigné);| \\où les paramètres sont des doubles.
\end{tabbing}


%\subsection{Image Finale}

%A compléter

\newpage%Include section Fonctionnement Statique

\section{Fonctionnement dynamique}
%SHADERS

\subsection{Principe du Shader}

Le fonctionnement dynamique du pipeline graphique se différencie du fonctionnement statique par l'utilisation de shaders.
Comme expliqué précédemment les shaders sonts de petits programmes écrits dans un langage spécifique : Avec OpenGl, le langage utilisé est le GLSL.\\
Ils sont compilés à l'éxécution du programme OpenGL, et executés pour chaque vertex (Vertex Shader) et chaque fragment (Fragment Shader).
Ils remplacent les calculs de matrices gérés nativement dans le fonctionnement statique d'OpenGL. C'est-à-dire qu'il faut implémenter ces calculs dans les shaders. L'intéret est qu'il est possible d'adapter ces calculs pour satisfaire certains besoins de l'application. D'autant plus que les shaders sont executés par la carte graphique(GPU) et que celle-ci, étant constituées d'une importante quantité de coeurs, peut facilement répartir les taches et faire tous ces calculs beaucoup plus rapidement que le processeur.

\subsection{Utilisation des shaders}

Il est plus ou moins aisé d'implémenter les shaders dans un programme OpenGL, suivant la bibliotèque graphique utilisée.
Nous prenons l'exemple de la SFML:\\

\begin{tabbing}
XXXX\=XXXX\= \kill\\
\> //On déclare un objet de type Shader\\
\> \verb|sf::Shader shader;|\\
\\
\>//Puis on charge les fichiers shaders\\
\> \verb|if(!shader.loadFromFile("vertex_shader.vert", sf::Shader::Vertex))|\\
\> \verb|{|\\
\> \>\verb|//error...|\\
\> \verb|}|\\
\end{tabbing}

Pour utiliser le shader il faut l'activer avant de dessiner et le desactiver quand on en a plus besoin.


\begin{tabbing}
XXXX\=XXXX\= \kill\\
\> // On active shader\\
\> \verb|sf::Shader::bind(&shader);|\\
\\
\> ....\\
\> //On dessine nos entités OpenGL ....\\
\> ....\\
\\
\> // On desactive le shader\\
\> \verb|sf::Shader::bind(NULL);|\\
\end{tabbing}


 %Include section Fonctionnement Dynamique

\section{Optimisations}
\subsection{Les VAO et VBO}

% ♥♥♥♥	♥♥♥♥
Les VBOs (Vertex Buffer Object) en OpenGL est une méthode qui permet d'envoyer des données vers la carte graphique, elle remplace les VA (Vertex Array) déclarés comme obsolètes par OpenGL qui étaient enregistrés sur le CPU et devaient transiter entre le CPU et le GPU.\\
Les VBOs ont été conçus afin d'obtenir les meilleurs performances possibles. Son utilisation est actuellement la méthode la plus efficace en OpenGL car les données 3D ne résident plus dans la mémoire système mais dans la mémoire de la carte graphique ce qui permet un rendu plus rapide.
\\\\
Les VAOs (Vertex Array Object) servent à optimiser l'utilisation des VBOs, ils ont une fonctionnalité toute nouvelle qui ressemble fortement à la Display List. Les VAOs permettent la "sauvegarde" de plusieurs commandes dans un seul et même objet qui lui même sera stocké dans la mémoire de la carte graphique. \\
Par exemple : ces commandes ou appels de fonctions peuvent utiliser des VBOs. Ainsi tout sera stocké directement dans la carte graphique et celle-ci n'aura plus à demander à l'application ce qu'elle doit faire. OpenGL pourra ainsi optimiser ses actions du faite qu'il connectera ses actions futurs. Ceci permet d'éviter de faire transiter trop d'informations entre le système et la carte graphique.
\\\\
Source : \cite{VBO}
\\\\
La création d’un VBO suit les étapes suivantes : 
\begin{itemize}
\item La génération du nouveau buffer object avec \verb|glGenBuffersARB ( );|
\item L’activation du buffer object avec \verb|glBindBufferARB ( );|
\item La copie de la donnée du vertex au buffer object avec \verb|glBufferDataARB ( );| 
\item L’usage du buffer pour le rendu de la donnée
\item La destruction du buffer
\end{itemize}
\subsubsection*{La fonction glGenBuffers} 
Créé le buffer object et retourne un nombre d’identifiants du buffer object. Deux paramètres sont attendus : 
\verb|void glGenBuffers(GLsizei n, GLuint *buffers);|
\\
Le premier, \verb|GLsizei n| où n renvoie au nombre de buffer object que l’on veut générer, et le second \verb|GLuint *buffers| qui renvoie les nombres identifiants de buffers dans un bloc mémoire commençant par buffers (élément simple ou tableau)

\subsubsection*{La fonction glBindBufferARB}

Une fois que le nom du buffer object est généré, il faut l’activer pour ensuite l’utiliser. La fonction utile est 
\begin{center}
\verb|void glBindBuffer(GLenum type, Gluint buffer) ;|
\end{center}
qui prend en compte deux paramètres. \verb|GLenum type| où type peut être \verb|GL_ARRAY_BUFFER| ou \verb|GL_ELEMENT_ARRAY_BUFFER| (d’autres types sont connus, comme \verb|GL_PIXEL_PACK_BUFFER| mais leur utilisation se fait avec le Pixel Buffer Object (PBO) que nous n’aborderons pas ici). \verb|GL_ARRAY_BUFFER| est utilisé quand le buffer object se réfère aux données des vertex (positions, couleurs, normales etc.) \verb|GL_ELEMENT_ARRAY_BUFFER| est utilisé quand les indices des vertex seront stockés dans le buffer

\subsubsection*{La fonction glBufferData}

A présent le buffer object est généré et prêt à recevoir des données. Sa taille est par défaut mise à zéro, la fonction \verb|glBufferDate| sert donc à initialiser le buffer avec les informations qui sont connues dans le tableau de données que l’on envoie.
\begin{center}
\verb|void glBufferData(GLenum type, GLsizeiptr size, const GLvoid * data, |\\
\verb|GLenum usage);|
\end{center}
\verb|GLenum| type est le même que celui du \verb|glBindBuffer()|, il peut prendre le paramètre \verb|GL_ARRAY_BUFFER| ou \verb|GL_ELEMENT_ARRAY_BUFFER|. \verb|GLsizeiptr size-| est la taille en bytes du tableau de vertex, \verb|const GlVoid *data| est le pointeur vers le tableau de données qui doit être copié. (Le tableau de position des vertex). Enfin usage fait référence à l’OpenGL et permet de définir l’utilisation du buffer (Voir Cas d’utilisation du buffer).
\subsubsection{Cas d'utilisation du buffer}

Le dernier paramètre de la fonction \verb|glBufferData()|, usage, est donc une indication sur la façon dont le buffer object va être utilisé . Usage fait référence à deux critères qui seront la fréquence à laquelle les informations seront modifiées et le type d’accès à ces données. Chaque constante sera la combinaison de ces deux critères et sera de la forme : \verb|GL_VALEURFREQUENCE_VALEURACCES| (cf. tableau 1 et tableau 2)
\\
\includegraphics[width=15cm,height=2.94cm]{img/tableau1.png}
\\
\includegraphics[width=15cm,height=3cm]{img/tableau2.png}

\subsubsection{Application d'un VA}
\includegraphics[width=15cm,height=7.48cm]{img/VA.png}
\subsubsection{Application d'un VBO}
\includegraphics[width=15cm,height=9.26cm]{img/VBO.png}
\subsubsection{Application d'un VBO dans un VAO}
\includegraphics[width=15cm,height=9.21cm]{img/VAO_VBO.png}
%♥♥♥♥	♥♥♥♥

%Include du fichier openGl.tex

\chapter{Bibliothèque de rendu 3D}
\input{bibliotheque_de_rendu_3d/bibliotheque_de_rendu_3d}

\chapter{MatLab}
% MATLABOUNET
\section{Introduction}
MatLab ou Matrix Laboratory est un logiciel développé par la société The MathWorks, sa fonctionnalité principale est d'effectuer des calculs numériques.\\ 
Il a initialement été conçu pour faciliter le traitement des matrices, cependant il est maintenant utilisé dans tous les domaines des sciences qui nécessite de faire des calculs (par exemple le traitement d'images).\\
Dans sa dernière version MatLab peut utiliser des fonctions d'accélération par le GPU.\\


\section{Les avantages et inconvénients}
\subsection{Avantages}
\begin{itemize}
\item Temps de programmation très rapide pour les calculs et l'affichage
\item Possibilité d'inclure un programme C ou C++
\item C'est un langage interprété, donc pas de compilation
\item Une librairie et une aide très bien faite et très riche
\item Code et fonctions facillement lisibles et compréhensibles
\end{itemize}

\subsection{Inconvénients}
\begin{itemize}
\item Vitesse de calcul beaucoup moins rapide qu'en C ou C++
\item Il est payant
% A vérifier 
\item Application auto-exécutable peu pratique
\end{itemize}

\section{Conclusion}
MatLab est donc généralement utilisé pour faire des expériences en peu de temps. Certaines applications qui nécessiteraient une journée de programmation en C ou C++ peuvent se réaliser en une heure sous MatLab. En revanche, une fois programmée, le temps de calcul sous MatLab peut être cent fois supérieur à celui de C ou C++. De ce fait, MatLab ne s'utilise que très peu pour réaliser un produit finit destiné à un particulier. 

\section{Sources}
http://www.nvidia.fr/object/tesla-matlab-accelerations-fr.html
%Include du fichier matlab.tex

%A compléter
\part{Problèmes Rencontrés}

\chapter{Optimisation}
\newpage
\end{document}