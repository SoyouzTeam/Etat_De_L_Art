\section{Exemple Structure code OpenGL}

\begin{tabbing}
XXXX\=XXXX\=XXXX\= \kill

\verb|int main(){|\\
\> \verb|//Création de la fenetre suivant la bibliotèque choisie|\\
\> \verb|//Le contexte OpenGL est défini à ce moment la par la bibliotèque|\\
\\	
\> \verb|glMatrixMode( GL_PROJECTION );//On active la matrice de Projection|\\
\> \verb|glLoadIdentity( );// On reinitialise la matrice actuelle (GL_PROJECTION)|\\
\> \verb|gluPerspective(80,(double)800/600,1,10);|\\
\> \verb|/*Définition de l'angle de vision de la caméra , du ratio de la fenetre, |\\
\> \verb|ainsi que de l'intervalle de profondeur dans lequel faire le rendu |\\
\> \verb|ici de 1 à 10.*/|\\
\\
\> \verb|glEnable(GL_DEPTH_TEST); //Activation du Z-Buffer|\\
\\
\> \verb|bool running = true;|\\
\\

\> \verb|while(running){|\\

\>\> \verb|glClear(GL_COLOR_BUFFER_BIT \GL_DEPTH_BUFFER_BIT);|\\
\>\> \verb|//Efface les buffers de couleurs ainsi que de Z-Buffer|\\
\\
\>\> \verb|glMatrixMode( GL_MODELVIEW );|\\
\>\> \verb|//On active la matrice ModelView|\\
\\
\>\> \verb|glLoadIdentity();// Réinitialise la matrice actuelle (GL_MODELVIEW)|\\
\\
\>\> \verb|gluLookAt(3,3,3,0,0,0,0,1,0);|\\
\>\> \verb|//On place la caméra en 3,3,3 dans le repère elle regarde|\\
\>\> \verb|l'origine du repère et la verticale est Y.|\\
\\
\>\> \verb|glBegin(GL_LINES);//On indique que l'on définit des lignes|\\
\>\>\> \verb|glColor3ub(0,0,255);//On change la couleur courante|\\
\>\>\> \verb|glVertex3i(0,0,0);//Ligne de ce point ....|\\
\>\>\> \verb|glVertex3i(1,0,0);//.... à ce point|\\
\>\>\> \verb|glColor3ub(0,255,0);//...|\\
\>\>\> \verb|glVertex3i(0,0,0);//...|\\
\>\>\> \verb|glVertex3i(0,1,0);//...|\\
\>\>\> \verb|glColor3ub(255,0,0);//...|\\
\>\>\> \verb|glVertex3i(0,0,0);//...|\\
\>\>\> \verb|glVertex3i(0,0,1);//...|\\
\>\> \verb|glEnd();// Fin de la définition des primitives|\\
\\
\>\> \verb|//Affichage à l'ecran en fonction de la bibliotèque|\\

\> \verb|}|\\
\> \verb|return 0;|\\
\verb|}|
\end{tabbing}

%BONJOUR
