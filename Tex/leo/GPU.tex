\section{Historique}
\begin{center}
\begin{tabular}{|c|c|m{0.2\linewidth}|m{0.3\linewidth} |c|}
\hline
Année & Génération & Carte & Application & Bus \\
\hline
1996 & 1 & 3dfx Voodoo & texture mapping et z-buffer & bus PCI\\
\hline
1998 & 2 & GeForce/ Radeon 7500 & GPU effectue Transform\&lighting, multi-texting & bus AGP \\
\cline{1-4}
2001 & 3 & GeForce3/ Radeon 8500 & Programmation sur les sommets (vertex shader)	& \\
\cline{1-4}
2002 & 4 & Radeon 9700/GeForce FX & Premières cartes programmables (fragment shader)	& \\
\hline
2008 & 5 & GeForce9/ Radeon HD & Compatibilité OpenGL et DirectX,  geometry shader & bus PCIe \\
\hline
\end{tabular}
\end{center}

Les bus PCI(Peripheral Component Interconnect), AGP(Advanced Graphics Port) et  PCIexpress sont des bus local.
Bus local : système de communication entre des cartes d’extension et la carte mère.
Le bus PCIe est une version plus petite et plus performante que le PCI et AGP.

\section{Composants}
\textbf{GPU} (Graphical Processing Unit) : Processeur graphique, constituant le cœur de la carte graphique, et qui possède des instructions évoluées de traitement d’image. \\\\
\textbf{Mémoire vidéo} (Frame Buffer) : Conserve les images traitées par le GPU avant l’affichage.

\section{Caractéristiques}
Un GPU est une unité de calcul massivement parallèle.
