\section{Fonctionnement dynamique}
%SHADERS

\subsection{Principe du Shader}

Le fonctionnement dynamique du pipeline graphique se différencie du fonctionnement statique par l'utilisation de shaders.
Comme expliqué précédemment les shaders sont de petits programmes écrits dans un langage spécifique : avec OpenGL, le langage utilisé est le GLSL.\\
Ils sont compilés à l'éxécution du programme OpenGL, et executés pour chaque vertex (Vertex Shader) et chaque pixel (Pixel Shader).
Ils remplacent les calculs de matrices gérés par défaut dans le fonctionnement statique d'OpenGL. C'est-à-dire qu'il faut implémenter ces calculs dans les shaders. L'intérêt est qu'il est possible d'adapter ces calculs pour satisfaire certains besoins de l'application. D'autant plus que les shaders sont exécutés par la carte graphique(GPU), qui est capable de répartir les opérations sur toutes les unités de calculs.

\subsection{Utilisation des shaders}

Il est plus ou moins aisé d'implémenter les shaders dans un programme OpenGL, suivant la bibliothèque graphique utilisée.
Nous prenons l'exemple de la SFML:\\

\begin{tabbing}
XXXX\=XXXX\= \kill\\
\> //On déclare un objet de type Shader\\
\> \verb|sf::Shader shader;|\\
\\
\>//Puis on charge les fichiers shaders\\
\> \verb|if (!shader.loadFromFile("vertex_shader.vert", "fragment_shader.frag"))|\\
\> \verb|{|\\
\> \>\verb|//error...|\\
\> \verb|}|\\
\end{tabbing}

Pour utiliser le shader il faut l'activer avant de dessiner et le désactiver quand on en a plus besoin.


\begin{tabbing}
XXXX\=XXXX\= \kill\\
\> // On active le shader\\
\> \verb|sf::Shader::bind(&shader);|\\
\\
\> ....\\
\> //On dessine nos entités OpenGL ....\\
\> ....\\
\\
\> // On désactive le shader\\
\> \verb|sf::Shader::bind(NULL);|\\
\end{tabbing}

\subsection{Exemple de Shader}
\subsubsection{Vertex Shader}

\begin{tabbing}
XXXX\=XXXX\= \kill\\
\> \verb|uniform float a;|\\
\> \\
\> \verb|void main(){|\\
\> \\
\>\> \verb|vec4 vertex = gl_Vertex;| \\
\> \\
\>\> \verb|vertex.y = vertex.y + a*cos(vertex.y)| \\
\> \\
\>\> \verb|gl_Position = gl_ModelViewProjectionMatrix * vertex;|\\
\> \\
\>\> \verb|// transforme les coordonnées de texture|\\
\>\> \verb|gl_TexCoord[0] = gl_TextureMatrix[0] * gl_MultiTexCoord0;|\\
\> \\
\>\> \verb|// transmet la couleur|\\
\>\> \verb|gl_FrontColor = gl_Color;|\\
\> \verb|}|\\
\end{tabbing}


\subsubsection{Pixel Shader}

\begin{tabbing}
XXXX\=XXXX\= \kill\\
\> \verb|uniform float a;| \\
\> \\
\> \verb|void main(){|\\
\> \\
\> \verb|gl_Position = gl_ModelViewProjectionMatrix * gl_Vertex;|\\
\> \\
\> \verb|}|\\
\end{tabbing}