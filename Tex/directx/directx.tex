\section{Introduction}
Direct3D est la partie affichage 2D et 3D de l'API propriétaire DirectX de Microsoft.Elle est disponible sur Windows, et est également la base de l'affichage sous Xbox. 
\newline
Il fournit un ensemble de fonctions permettant d'afficher en 2D ou 3D.
\newline
Contrairement à OpenGL, Microsoft fournit son unique implementation de Direct3D à tous les constructeurs, qui eux n'ont plus qu'à l'implementer dans les drivers de leurs matériels.
  
\section{Les avantages et inconvénients}
\subsection{Avantages}
\begin{itemize}
\item Plus connu et utilisé qu'OpenGL.
\item Une communauté plus importante et donc un meilleur support.
\item Direct3D est une composante de l'API DirectX qui englobe nombreux autres modules ( Réseau, Son, Entrée/Sortie, ...)
\end{itemize}

\subsection{Inconvénients}
\begin{itemize}
\item Il est exclusif à l'environnement Windows.
\item Les fonctionnalités Direct3D évoluent en fonction des nouvelles versions de Windows(Seven, 8) et sont peu compatibles avec les anciennes telle que XP. 
\item Moins rapide qu'OpenGL dans les calculs.
\end{itemize}

\section{Conclusion}
Direct3D est très representé, notamment dans le domaine du "visuel", des jeux vidéos. Ceci étant du au fait que DirectX est très complet et cohérent par rapport à ce type d'activité. OpenGL, quant à lui, est portable, plus puissant. En considérant l'importante masse de données à gérer dans notre programme l'usage d'OpenGL est évident d'autant plus qu'OpenGL est plus facile à prendre en main que Direct3D. 