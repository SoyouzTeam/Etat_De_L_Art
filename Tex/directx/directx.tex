\section{Introduction}
Direct3D est la partie affichage 2D et 3D de l'API propriétaire DirectX de Microsoft.Elle est disponible sur Windows, et est également la base de l'affichage sous Xbox. 
\newline
Elle fournit un ensemble de fonctions permettant d'afficher en 2D ou 3D.
\newline
Contrairement à OpenGL, Microsoft fournit son implementation de Direct3D à tous les constructeurs, qui eux n'ont plus qu'à l'utiliser dans le driver de leurs matériel.
  
\section{Les avantages et inconvénients}
\subsection{Avantages}
\begin{itemize}
\item Plus connu et utilisé qu'OpenGL.
\item Une aide très large.
\item Une librairie très riche.
\item Calcul de données très rapide.
\end{itemize}

\subsection{Inconvénients}
\begin{itemize}
\item Il est payant.
\item Les fonctionnalités Direct3D évoluent en fonction des nouvelles versions de Windows (seven,8) et sont peu compatibles avec les anciennes tel que XP. 
\item Reconnu moins rapide qu'OpenGL.
\item Direct3D n'est implémenté que sur Windows (non multi platforme).
\end{itemize}

\section{Conclusion}
Le seul fait que l'API soit payante est déjà assez contraignante, mais le fait qu'OpenGL soit plus rapide, est l'argument décisif dans le choix de la bibliothèque pour notre projet. En effet, en raison de la masse de données à gérer dans notre programme, il est préférable d'utiliser OpenGL.