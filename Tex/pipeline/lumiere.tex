\section{Généralité}
Une partie de la lumière vient d’une direction ou d’une position particulière, alors
qu’une autre partie est généralement dispersée sur l’ensemble de la scène, c'est pourquoi, le modèle d'éclairage peut être divisé en plusieurs composantes : 
\begin{itemize}
\item Lumière ambiante
\item Lumière diffuse
\item Lumière spéculaire
\item Lumière émissive
\end{itemize}

\section{Lumière ambiante}
Cette composante est la lumière qui ne semble pas avoir de source précise, elle provient de tous les point de la scène.
\\\\
Elle ne permet pas à elle seule de voir les détails des objets de la scène car elle éclaire toute la scène de la même façon, sans ombre.

\section{Lumière diffuse}
Cette composante est la lumière qui provient d'une source de lumière précise, telle qu'une lampe, ou le soleil.
\\\\
Une lumière est diffuse quand elle est réfléchie par les objets qu'elle éclaire.
\\\\
Elle apparait donc brillante quand elle touche un objet, et est réfléchie à partir de l'objet dans toutes les direction de la scène.

\section{Lumière spéculaire}
Cette composante est la lumière qui arrive d'une direction et qui rebondit sur la surface en fonction des propriétés de l'objet et de son orientation par rapport à la source.

\section{Lumière émissive}
Cette composante est la lumière provenant d'un objet.
\\\\
Elle permet d'ajouter de l'intensité à un objet.
\\\\
Elle n'affecte pas et n'est pas affecté par les autres composantes.